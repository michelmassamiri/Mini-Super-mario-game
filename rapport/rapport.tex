\documentclass{article}

\usepackage[utf8]{inputenc}    
\usepackage[T1]{fontenc}
\usepackage[francais]{babel}     


\title{Rapport du projet de programmation systeme}
\date{06/01/2017}
\author{Demoulins Louis \and Massamiri Michel}

\begin{document}


\maketitle


\newpage
\section{Introduction}

Le présent projet a été réalisé dans le cadre de l'unité d'enseignement Programmation Système de l'université de Bordeaux. Il consiste à creer une fonction de sauvegarde et de gestions de celles ci, ainsi que de la gestion de temporisateurs au sein d'un jeu.

Le système de sauvegarde doit être en mesure de sauvegarder les objets que l'on place sur la carte et de pouvoir les recharger après tel quel. Ainsi on peut creer plusieurs niveaux dans le jeu sans avoir à recommencer la création de la carte à chaque lancement du jeu.

Le système de gestion de sauvegarde à pour but d'avoir des informations sur les cartes sauvegardées (connaitre leurs hauteur, largeurs, nombre d'objets différents) ainsi que de faires quelques modifications.

La gestion des temporisateur est utilisé pour pouvoir gerer des éléments du jeu qui sont activé et ont une action diféré dans le jeu (exemple : la bombe se lance et explose deux secondes plus


\newpage
\section{Deuxième partie: Gestion des temporisateurs}
\end{document}

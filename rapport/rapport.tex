\documentclass{article}

\usepackage[utf8]{inputenc}    
\usepackage[T1]{fontenc}
\usepackage[francais]{babel}     


\title{Rapport du projet de programmation systeme}
\date{06/01/2017}
\author{Demoulins Louis \and Massamiri Michel}

\begin{document}


\maketitle


\newpage
\section{Introduction}

Le présent projet a été réalisé dans le cadre de l'unité d'enseignement Programmation Système de l'université de Bordeaux. Il consiste à creer une fonction de sauvegarde et de gestions de celles ci, ainsi que de la gestion de temporisateurs au sein d'un jeu.

Le système de sauvegarde doit être en mesure de sauvegarder les objets que l'on place sur la carte et de pouvoir les recharger après tel quel. Ainsi on peut creer plusieurs niveaux dans le jeu sans avoir à recommencer la création de la carte à chaque lancement du jeu.

Le système de gestion de sauvegarde à pour but d'avoir des informations sur les cartes sauvegardées (connaitre leurs hauteur, largeurs, nombre d'objets différents) ainsi que de faires quelques modifications.

La gestion des temporisateur est utilisé pour pouvoir gerer des éléments du jeu qui sont activé et ont une action diféré dans le jeu (exemple : la bombe se lance et explose deux secondes plus)

\newpage
\section{Première partie: Gestion des Sauvegardes}

\subsection{Sauvegarde}
Pour la sauvegarde nous avons principalement utilisé les fonctions que nous donne la bibliothèque du jeu pour récupérer les éléments de la carte (sa hauteur, largeur, le nombre d'objet, quel objets est placé en [x;y], est-ce que cet objet est destructible, etc.), ainsi que les appels systèmes permettant l'écriture dans les fichiers. 

Nous avons commencer par creer un fichier nommé arbitrairement "save.map", dans lequel nous avons ensuite inscrit la hauteur et la largeur de la carte, suivit du nombre d'objets différents que l'on va y trouver. Ensuite nous avons ajouter tous les éléments de la carte sous la forme de leur identifiants nulmérique, puis, finalement, nous y avons intégrer les définitions des objets.

\subsection{Chargement}
En ce qui concerne le chargement nous avons utilisé la méthode oposé au chargement : au lieu d'écrire dans le fichiers les éléments du jeux un par un, on les lit dans le fichier spécifié, pour on les intègre au jeu grâce aux différentes fonctions qui sont proposé par la bibliothèque du jeu.

\subsection{Maputil}
Maputil est un utilitaire permettant de récuperrer les informations d'une carte depuis son fichier de sauvegarde, ainsi que de modifier certains éléments de la sauvegarde.

\subsubsection{Eléments get}
Pour les éléments "get" (c'est à dire les options getwidth, getheight, getobjects et getinfo) a programmer, nous avons simplement fait une lecture du fichier, et, avec un jeu de conditions, afficher uniquement les éléments qui nous interessent. Cette partie est donc très similaire au système de chargement du jeu, mis à part qu'on ne fait qu'afficher à l'écran ce qu'on lit, on ne crée pas d'objets utilisables en jeu.

\subsubsection{Elément set}
Pour les options setwidth et setheight il a fallut voir plusieurs cas de figure. Mais avant tout il faut transformer le fichier de sauvegarde en un fichier temporaire "tmp", puis recreer le fichier de sauvegarde. Dans le nouveau fichier on insère les nouvelles valeurs de largeur et hauteur, ainsi que les informations relatives au objets. Ensuite si la carte est plus grande on ajoute juste des 0 la ou il en manque.

Dans le cas ou la carte est plus petite il n'y a qu'a lire les données et bien faire attention a sauter les données de l'ancienne carte.

On fini par recopier la fin du fichier, puis on supprime le fichier tmp qui ne nous sert plus à rien.

\subsubsection{Remplacement d'objets}
Pour remplacer les éléments du jeu, on commence par faire la même chose que pour changer la taille, on crée le fichier tmp et on recrée un fichier avec le même bom que la sauvegarde. Ensuite on parcourt le fichier tmp jusqu'a l'endroit ou sont stockés les objets du jeu tout en recopiant les données dans le fichier, puis on insère les données passées en paramètre dans le nouveau fichier.


\subsubsection{"Prune Objects"}
Dans le but d'enlever des objets inutiles de la sauvegarde, il a été demandé d'implémenter une fonction de "prune objects". Pour ce faire il faut parcour la carte en notant tous les objets utilisé dans un tableau. Si par exemple seuls les objets 3 et 5 sont utilisé, le tableau aura la forme suivante : t = {3, 5}.

Une fois que l'on a isolé les objets que l'on utilise de ceux qui sont inutilisés, il suffit de creer un nouveau fichier (la méthode du tmp qu'on a déjà utiliser au dessus), de recopier le fichier jusqu'à ce qu'on arrive à la partie des objets, puis on saute les objets inutilisé, c'est à dire, si les id des objets ne sont pas dans le tableau t, on ne les recopies pas. Etant donné que les objets sont classé par ordre numérique, il suffite de recopier, dans ce cas précis par exemple, le 3eme et le 5eme objet.


\newpage
\section{Deuxième partie: Gestion des temporisateurs}

Pour cette partie, on avait travaillé dans le fichier source «tempo.c» qui contient les fonctions: 

‘timer_init’ ayant pour mission, l’initialisation de toutes les variables dont on aura besoin dans la suite, ainsi de créer un ‘thread’ qui doit être le seul thread recevant le signal SIGALRM. 

‘timer_set’ qui permet d’armer les temporisateurs avec des différents délais.
\subsection{Démon récepteur de signaux}

Afin de pouvoir lancer le thread Démon qui devrait être le seul à recevoir SIGALRM, on avait configuré pour cette partie un temporisateur ayant un délai pour deux seconds, ensuite le thread est lancé avec NULL comme des paramètres.

Dans le thread Démon, on initialise le traitant du signal SIGALRM qui est la fonction :

void handler(int sig)

En effet, chaque thread hérite du processus principal le masque des signaux. Donc pour bloquer SIGALRM dans tous les autres threads, il suffit de définir un masque de signaux qu’on lui ajoute SIGALRM et donc bloquer tous les autres threads qui vont être lancés après Démon.

Ensuite, le thread Démon sera bloqué en utilisant ‘sigsuspend’ tant qu’il reçoit pas un SIGALRM.
Dès que Démon reçoit le signal SIGALRM, il appelle la fonction ‘handler’ étant le traitant qui devrait effectivement afficher le numéro du thread qui l’a appelé.

\subsection{Implémentation simple}

En considérant qu’il n’y a qu’un seul temporisateur armé à tout instant, donc on n’a pas besoin de gérer et de maintenir l’échéancier complet. 

Tout d’abord, dans la fonction ‘timer_set’, on initialise notre temporisateur avec les valeurs:

struct itimerval timer ;

timer.it_value.tv_sec = 0 ;
timer.it_value.tv_usec = 0 ;
timer.it_interval.tv_sec = 0 ;
timer.it_interval.tv_usec = 0 ;

Et afin d’envoyer le signal SIGALRM au bon moment, une boucle ‘while(delay ≥ 1000)’ était nécessaire. Donc, chaque fois qu’on rentre dans la boucle on souscrit 1000 de ‘delay’, et on incrémente  ‘timer.it_value.tv_sec’ par 1. Au final, ‘timer.it_value.tv_sec’ est égale à ‘delay * 1000’.
Ce qui nous permet d’envoyer le signal SIGALRM dans le délai passé en paramètre.

Alors pour pouvoir appeler la fonction ‘sdl_push_event’ avec la bonne paramètre ‘param’, on avait ajouté une variable globale dans le fichier source «tempo.c» qui à chaque fois qu’on appelle timer_set, la variable globale ‘PARAM’ sauvegarde la nouvelle valeur de param passée en paramètre dans la fonction timer_set.

Au final, on appelle l’instruction ‘printf’ dans la fonction traitant avec la variable ‘PARAM’.

\end{document}
